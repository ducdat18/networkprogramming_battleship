\subsection{Serialization \& Wire Format Examples}

This section explains how message structs are serialized to binary format for network transmission and provides real-world examples.

\subsubsection{Serialization Mechanism}

The protocol uses \textbf{binary serialization} with packed C structs. This approach directly maps memory structures to network bytes without intermediate encoding (unlike JSON or XML).

\begin{figure}[h]
\centering
\includegraphics[width=\textwidth]{images/serialization_process.png}
\caption{Complete Serialization and Deserialization Process}
\label{fig:serialization_process}
\end{figure}

\paragraph{Serialization Approach}

The protocol uses **direct memory mapping** for serialization:

\begin{itemize}
    \item \textbf{Serialize}: Cast struct pointer to \texttt{char*}, copy \texttt{sizeof(T)} bytes
    \item \textbf{Deserialize}: Validate size, then \texttt{memcpy} into struct
    \item \textbf{Zero overhead}: No parsing, no intermediate representation
    \item \textbf{Type-safe}: Template functions ensure correct struct types
\end{itemize}

\textbf{Example Flow:}
\begin{verbatim}
LoginRequest struct (in memory)
    ↓ serialize()
Binary string (96 bytes)
    ↓ send()
Network transmission
    ↓ recv()
Binary string (96 bytes)
    ↓ deserialize()
LoginRequest struct (in memory)
\end{verbatim}

\subsubsection{Packed Structs}

The \texttt{\_\_attribute\_\_((packed))} directive is \textbf{critical} for binary compatibility:

\begin{table}[h]
\centering
\begin{tabular}{|l|c|c|}
\hline
\textbf{Struct} & \textbf{Without Packing} & \textbf{With Packing} \\ \hline
Size & 8 bytes (padding added) & 5 bytes (exact) \\ \hline
Compatibility & ✗ Platform-dependent & ✓ Consistent \\ \hline
Network & ✗ Incompatible & ✓ Works \\ \hline
\end{tabular}
\caption{Impact of Struct Packing}
\end{table}

Without packing, compilers add alignment padding which varies by architecture (x86, ARM, etc.), causing deserialization errors.

\subsubsection{Hex Dump Example: LoginRequest}

\textbf{LoginRequest Structure:}
\begin{lstlisting}[style=cppstyle]
LoginRequest {
    username: "alice",
    password: "5f4dcc3b5aa765d61d8327deb882cf99"
              "2af9ce19b66abbed229bdeb6af04d29c"
}
\end{lstlisting}

\textbf{Binary Representation (96 bytes):}

\begin{verbatim}
Offset(h) 00 01 02 03 04 05 06 07 08 09 0A 0B 0C 0D 0E 0F  ASCII
------------------------------------------------------------------------
00000000  61 6c 69 63 65 00 00 00 00 00 00 00 00 00 00 00  alice...........
00000010  00 00 00 00 00 00 00 00 00 00 00 00 00 00 00 00  ................
00000020  35 66 34 64 63 63 33 62 35 61 61 37 36 35 64 36  5f4dcc3b5aa765d6
00000030  31 64 38 33 32 37 64 65 62 38 38 32 63 66 39 39  1d8327deb882cf99
00000040  32 61 66 39 63 65 31 39 62 36 36 61 62 62 65 64  2af9ce19b66abbed
00000050  32 32 39 62 64 65 62 36 61 66 30 34 64 32 39 63  229bdeb6af04d29c
------------------------------------------------------------------------
Total: 96 bytes (0x60)
\end{verbatim}

\textbf{Analysis:}
\begin{itemize}
    \item \textbf{Offset 0x00-0x1F (32 bytes):} Username field "alice" + null padding
    \item \textbf{Offset 0x20-0x5F (64 bytes):} SHA-256 hash as hex string
\end{itemize}

\subsubsection{Hex Dump Example: MoveMessage}

\textbf{MoveMessage Structure:}
\begin{lstlisting}[style=cppstyle]
MoveMessage {
    match_id: 123 (0x0000007B),
    target: {row: 5, col: 7}
}
\end{lstlisting}

\textbf{Binary Representation (6 bytes, little-endian):}

\begin{verbatim}
Offset(h) 00 01 02 03 04 05  Interpretation
-------------------------------------------------
00000000  7B 00 00 00 05 07  |{.....  |
          |__________|  |  |
          match_id(4)  row col
\end{verbatim}

\textbf{Breakdown:}
\begin{itemize}
    \item \texttt{0x7B 00 00 00}: match\_id = 123 (uint32\_t, little-endian)
    \item \texttt{0x05}: row = 5 (int8\_t)
    \item \texttt{0x07}: col = 7 (int8\_t)
\end{itemize}

\subsubsection{Complete Message Example}

\textbf{Full network transmission of a MOVE message:}

\begin{verbatim}
+------------------+------------------+
| MessageHeader    | Payload          |
| (77 bytes)       | (6 bytes)        |
+------------------+------------------+

MessageHeader (77 bytes):
  type:           0x1F (31 = MessageType::MOVE)
  length:         0x06 00 00 00 (6 bytes)
  timestamp:      0x6B 8A 9C 65 00 00 00 00 (Unix time: 1703812715)
  session_token:  "token_42_a7f3e9d2c1b8f4a6e3d7c9b2a5f1e8d4..."

Payload (6 bytes):
  match_id:       0x7B 00 00 00 (123)
  target.row:     0x05 (5)
  target.col:     0x07 (7)

Total: 83 bytes sent over TCP socket
\end{verbatim}

\subsubsection{Endianness}

The protocol uses \textbf{little-endian} byte order (x86/x64 standard):

\begin{table}[h]
\centering
\begin{tabular}{|l|l|}
\hline
\textbf{Value} & \textbf{Memory Layout (Little-Endian)} \\ \hline
0x12345678 & 78 56 34 12 \\ \hline
123 (0x0000007B) & 7B 00 00 00 \\ \hline
\end{tabular}
\caption{Little-Endian Byte Order}
\end{table}

\textbf{Note:} Since both client and server run on x86/x64 Linux, no byte-order conversion is needed. For cross-platform support (ARM, MIPS), functions like \texttt{htonl/ntohl} would be required.

\subsubsection{Size Validation}

All deserialization includes strict size validation:

\begin{itemize}
    \item \textbf{Exact match required}: \texttt{payload.size() == sizeof(T)}
    \item \textbf{Reject if incomplete}: Prevents buffer underflows
    \item \textbf{Reject if too large}: Prevents buffer overflows
\end{itemize}

This prevents malformed or malicious messages from corrupting system state.

\subsubsection{Advantages of Binary Serialization}

\begin{table}[h]
\centering
\begin{tabular}{|l|c|c|}
\hline
\textbf{Feature} & \textbf{Binary} & \textbf{JSON} \\ \hline
Message Size & Small (96 bytes) & Large (~200 bytes) \\ \hline
Parsing Speed & Fast (memcpy) & Slow (parsing) \\ \hline
Type Safety & Strong (compiled) & Weak (runtime) \\ \hline
Human Readable & No & Yes \\ \hline
Debugging & Hex dumps & Easy inspection \\ \hline
Bandwidth & Efficient & Wasteful \\ \hline
\end{tabular}
\caption{Binary vs JSON Serialization Comparison}
\label{tab:serialization_comparison}
\end{table}

For a real-time game where performance matters, binary serialization is the superior choice despite reduced human readability.
