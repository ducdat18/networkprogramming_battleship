\subsection{Protocol Overview}

The Battleship multiplayer game uses a custom binary protocol built on top of TCP sockets for reliable client-server communication. The protocol is designed for efficiency, security, and ease of implementation.

\subsubsection{Protocol Characteristics}

\begin{itemize}
    \item \textbf{Transport Layer}: TCP (Transmission Control Protocol)
    \item \textbf{Socket API}: POSIX sockets (BSD-style)
    \item \textbf{Protocol Version}: 1.0
    \item \textbf{Default Port}: 8888
    \item \textbf{Message Format}: Binary (packed C structs)
    \item \textbf{Encoding}: Little-endian (x86/x64 compatible)
\end{itemize}

\subsubsection{Protocol Constants}

The protocol defines several important constants used throughout the implementation:

\begin{table}[h]
\centering
\begin{tabular}{|l|l|l|}
\hline
\textbf{Constant} & \textbf{Value} & \textbf{Description} \\ \hline
PROTOCOL\_VERSION & "1.0" & Current protocol version \\ \hline
DEFAULT\_PORT & 8888 & Server listening port \\ \hline
MAX\_MESSAGE\_SIZE & 4096 bytes & Maximum single message size \\ \hline
BUFFER\_SIZE & 8192 bytes & Network buffer size \\ \hline
SESSION\_EXPIRY & 24 hours & Session timeout duration \\ \hline
CHALLENGE\_TIMEOUT & 60 seconds & Challenge expiration time \\ \hline
\end{tabular}
\caption{Protocol Constants}
\label{tab:protocol_constants}
\end{table}

\subsubsection{Message Structure}

Each network message follows a consistent structure:

\begin{equation}
\text{Message} = \text{MessageHeader (77 bytes)} + \text{Payload (variable)}
\end{equation}

The fixed-size header allows for efficient message parsing and routing, while the variable-length payload supports different message types with varying data requirements.

\subsubsection{Wire Format}

Each message consists of a fixed-size header followed by a variable-length payload:

\begin{figure}[h]
\centering
\includegraphics[width=0.9\textwidth]{images/wire_format.png}
\caption{Message Wire Format - Header and Payload Structure}
\label{fig:wire_format}
\end{figure}

\subsubsection{Design Rationale}

\textbf{Why Binary Protocol?}
\begin{itemize}
    \item \textbf{Efficiency}: Binary encoding is more compact than text-based protocols (JSON/XML)
    \item \textbf{Performance}: No serialization/deserialization overhead
    \item \textbf{Type Safety}: Fixed-size structs prevent parsing errors
    \item \textbf{Simplicity}: Direct memory mapping with \texttt{memcpy}
\end{itemize}

\textbf{Why TCP over UDP?}
\begin{itemize}
    \item \textbf{Reliability}: Guaranteed delivery and ordering (critical for game state)
    \item \textbf{Error Handling}: Built-in checksums and retransmission
    \item \textbf{Simplicity}: No need to implement custom acknowledgment logic
    \item \textbf{Firewall-Friendly}: Better NAT traversal than UDP
\end{itemize}

\subsubsection{Security Considerations}

\begin{itemize}
    \item \textbf{Session Tokens}: 64-byte tokens for authentication (included in every message header)
    \item \textbf{Password Hashing}: SHA-256 with salt (passwords never transmitted in plaintext)
    \item \textbf{Size Validation}: Payload size checked before deserialization (prevents buffer overflows)
    \item \textbf{Timeout Protection}: Session and challenge timeouts prevent stale connections
\end{itemize}
