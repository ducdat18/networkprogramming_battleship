\subsection{Phase 2: Authentication System}

The authentication system handles user registration, login, session management, and auto-login functionality with strong security measures.

\subsubsection{User Registration}

\paragraph{Registration UI}

\begin{figure}[h]
\centering
\includegraphics[width=0.7\textwidth]{images/registration_screen.png}
\caption{User Registration Screen}
\label{fig:registration_screen}
\end{figure}

\paragraph{Validation Rules}

\begin{table}[h]
\centering
\begin{tabular}{|l|l|}
\hline
\textbf{Field} & \textbf{Validation} \\ \hline
Username & 4-20 characters, alphanumeric only, must be unique \\ \hline
Password & Minimum 6 characters, any characters allowed \\ \hline
Display Name & 2-30 characters, UTF-8 allowed \\ \hline
Email & Optional, valid email format if provided \\ \hline
\end{tabular}
\caption{Registration Field Validation}
\label{tab:registration_validation}
\end{table}

\paragraph{Registration Process}

\textbf{Server validates and creates user:}
\begin{enumerate}
    \item Validate username (4-20 chars, unique)
    \item Validate display name (2-30 chars)
    \item Hash password with salt (SHA-256)
    \item Insert into database with initial ELO=1000
    \item Return success with user\_id or error message
\end{enumerate}

\subsubsection{Password Security}

\paragraph{Password Hashing Flow}

The password is hashed twice for security:

\begin{enumerate}
    \item \textbf{Client-side}: SHA-256(password) → 64-char hex string
    \item \textbf{Server-side}: SHA-256(salt + client\_hash) → stored as \texttt{salt\$hash}
\end{enumerate}

\paragraph{Client-Side Hashing}

Located in \texttt{client/src/password\_hash.cpp}:

\begin{lstlisting}[style=cppstyle]
#include <openssl/sha.h>

std::string hashPasswordClient(const std::string& password) {
    // SHA-256 hash
    unsigned char hash[SHA256_DIGEST_LENGTH];
    SHA256_CTX sha256;
    SHA256_Init(&sha256);
    SHA256_Update(&sha256, password.c_str(), password.size());
    SHA256_Final(hash, &sha256);

    // Convert to hex string (64 characters)
    char hex_string[65];
    for (int i = 0; i < SHA256_DIGEST_LENGTH; i++) {
        sprintf(&hex_string[i*2], "%02x", hash[i]);
    }
    hex_string[64] = '\0';

    return std::string(hex_string);
}
\end{lstlisting}

\paragraph{Server-Side Salted Hashing}

Located in \texttt{common/src/password\_hash.cpp}:

\begin{lstlisting}[style=cppstyle]
#include <openssl/rand.h>
#include <openssl/sha.h>

std::string hashPassword(const std::string& password_hash) {
    // Generate random 16-byte salt
    unsigned char salt[16];
    if (RAND_bytes(salt, 16) != 1) {
        throw std::runtime_error("Failed to generate salt");
    }

    // Hash: SHA256(salt + password_hash)
    unsigned char final_hash[SHA256_DIGEST_LENGTH];
    SHA256_CTX sha256;
    SHA256_Init(&sha256);
    SHA256_Update(&sha256, salt, 16);
    SHA256_Update(&sha256, password_hash.c_str(),
                  password_hash.size());
    SHA256_Final(final_hash, &sha256);

    // Convert to hex
    std::string salt_hex = hexEncode(salt, 16);
    std::string hash_hex = hexEncode(final_hash,
                                     SHA256_DIGEST_LENGTH);

    // Store as: salt$hash
    return salt_hex + "$" + hash_hex;
}

bool verifyPassword(const std::string& password_hash,
                    const std::string& stored_hash) {
    // Split stored_hash into salt and hash
    size_t pos = stored_hash.find('$');
    if (pos == std::string::npos) {
        return false;
    }

    std::string salt_hex = stored_hash.substr(0, pos);
    std::string expected_hash_hex = stored_hash.substr(pos + 1);

    // Decode salt from hex
    unsigned char salt[16];
    hexDecode(salt_hex, salt, 16);

    // Recompute hash with same salt
    unsigned char computed_hash[SHA256_DIGEST_LENGTH];
    SHA256_CTX sha256;
    SHA256_Init(&sha256);
    SHA256_Update(&sha256, salt, 16);
    SHA256_Update(&sha256, password_hash.c_str(),
                  password_hash.size());
    SHA256_Final(computed_hash, &sha256);

    std::string computed_hex = hexEncode(computed_hash,
                                         SHA256_DIGEST_LENGTH);

    // Constant-time comparison (prevent timing attacks)
    return computed_hex == expected_hash_hex;
}
\end{lstlisting}

\subsubsection{Login \& Session Management}

\paragraph{Login UI}

\begin{figure}[h]
\centering
\includegraphics[width=0.7\textwidth]{images/login_screen.png}
\caption{Login Screen}
\label{fig:login_screen}
\end{figure}

\paragraph{Login Flow}

\begin{enumerate}
    \item User enters username and password
    \item Client hashes password with SHA-256
    \item Client sends \texttt{AUTH\_LOGIN} with hashed password
    \item Server validates credentials against database
    \item Server generates unique session token
    \item Server stores session in database (24h expiration)
    \item Server sends \texttt{AUTH\_RESPONSE} with session token and user data
    \item Client stores session token for subsequent requests
\end{enumerate}

\paragraph{Session Token Generation}

\begin{lstlisting}[style=cppstyle]
std::string generateSessionToken(uint32_t user_id) {
    // Generate random 32-byte token
    unsigned char random_bytes[32];
    RAND_bytes(random_bytes, 32);
    std::string random_hex = hexEncode(random_bytes, 32);

    // Format: token_{user_id}_{random_hash}
    return "token_" + std::to_string(user_id) +
           "_" + random_hex.substr(0, 32);
}
\end{lstlisting}

\paragraph{Session Storage}

Database schema in \texttt{server/src/database.cpp}:

\begin{lstlisting}[style=sqlstyle]
CREATE TABLE sessions (
    session_id INTEGER PRIMARY KEY AUTOINCREMENT,
    user_id INTEGER NOT NULL,
    session_token TEXT UNIQUE NOT NULL,
    created_at DATETIME DEFAULT CURRENT_TIMESTAMP,
    expires_at DATETIME NOT NULL,
    FOREIGN KEY (user_id) REFERENCES users(user_id)
        ON DELETE CASCADE
);

-- Index for fast token lookup
CREATE INDEX idx_session_token ON sessions(session_token);

-- Index for expiration cleanup
CREATE INDEX idx_expires_at ON sessions(expires_at);
\end{lstlisting}

\paragraph{Session Validation}

\begin{lstlisting}[style=cppstyle]
bool DatabaseManager::validateSession(
    const std::string& session_token,
    uint32_t& user_id
) {
    const char* sql =
        "SELECT user_id FROM sessions "
        "WHERE session_token = ? "
        "AND expires_at > datetime('now')";

    sqlite3_stmt* stmt;
    sqlite3_prepare_v2(db_, sql, -1, &stmt, nullptr);
    sqlite3_bind_text(stmt, 1, session_token.c_str(), -1,
                      SQLITE_STATIC);

    bool valid = false;
    if (sqlite3_step(stmt) == SQLITE_ROW) {
        user_id = sqlite3_column_int(stmt, 0);
        valid = true;
    }

    sqlite3_finalize(stmt);
    return valid;
}
\end{lstlisting}

\subsubsection{Auto-Login Feature}

\paragraph{Session Persistence}

Located in \texttt{client/src/session\_storage.cpp}:

\begin{lstlisting}[style=cppstyle]
class SessionStorage {
private:
    std::string getSessionFilePath() {
        const char* home = getenv("HOME");
        if (!home) {
            throw std::runtime_error("HOME not set");
        }

        std::string dir = std::string(home) + "/.battleship";

        // Create directory if not exists
        mkdir(dir.c_str(), 0700);

        return dir + "/session.txt";
    }

public:
    void saveSession(const std::string& session_token) {
        std::string path = getSessionFilePath();
        std::ofstream file(path, std::ios::trunc);

        if (!file.is_open()) {
            std::cerr << "Failed to save session\n";
            return;
        }

        file << session_token;
        file.close();

        // Set restrictive permissions (owner read/write only)
        chmod(path.c_str(), 0600);
    }

    std::string loadSession() {
        std::string path = getSessionFilePath();
        std::ifstream file(path);

        if (!file.is_open()) {
            return "";  // No saved session
        }

        std::string token;
        std::getline(file, token);
        file.close();

        return token;
    }

    void clearSession() {
        std::string path = getSessionFilePath();
        unlink(path.c_str());
    }
};
\end{lstlisting}

\paragraph{Auto-Login Flow}

\begin{enumerate}
    \item On application startup, check for saved session file
    \item If found, load session token
    \item Send \texttt{VALIDATE\_SESSION} request to server
    \item If valid, skip login screen and proceed to lobby
    \item If invalid/expired, show login screen and delete saved session
\end{enumerate}

\begin{lstlisting}[style=cppstyle]
void UIManager::attemptAutoLogin() {
    std::string token = session_storage_.loadSession();

    if (token.empty()) {
        showLoginScreen();
        return;
    }

    // Validate session with server
    SessionValidateRequest req;
    strncpy(req.session_token, token.c_str(), 63);
    req.session_token[63] = '\0';

    std::string payload = serialize(req);
    network_->sendMessage(MessageType::VALIDATE_SESSION,
                          payload);

    // Wait for response (timeout: 5 seconds)
    // If valid → showLobby()
    // If invalid → showLoginScreen() + clearSession()
}
\end{lstlisting}

\subsubsection{Database Schema}

\paragraph{Users Table}

\begin{lstlisting}[style=sqlstyle]
CREATE TABLE users (
    user_id INTEGER PRIMARY KEY AUTOINCREMENT,
    username TEXT UNIQUE NOT NULL,
    password_hash TEXT NOT NULL,  -- Format: salt$hash
    display_name TEXT NOT NULL,
    email TEXT,
    elo_rating INTEGER DEFAULT 1000,
    created_at DATETIME DEFAULT CURRENT_TIMESTAMP,
    last_login DATETIME
);

CREATE INDEX idx_username ON users(username);
\end{lstlisting}

% Database ERD belongs to Person 1 (Database Design section)

\subsubsection{Security Considerations}

\begin{itemize}
    \item \textbf{Double Hashing}: Client SHA-256 + Server SHA-256 with salt
    \item \textbf{Salt Randomness}: 16 bytes from OpenSSL RAND\_bytes (cryptographically secure)
    \item \textbf{Session Expiration}: 24-hour timeout prevents indefinite sessions
    \item \textbf{File Permissions}: Session file set to 0600 (owner-only)
    \item \textbf{No Plaintext}: Passwords never transmitted or stored in plaintext
    \item \textbf{SQL Injection}: Parameterized queries with prepared statements
\end{itemize}
